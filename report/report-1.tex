\documentclass[]{IEEEtran}
\usepackage[utf8]{inputenc}
\usepackage{listings}

\author{Rodrigo Lourenço, ist176133, group 9}
\title{Development of a simple RISC processor}

\begin{document}



\maketitle



\section{Introduction}

The goal for this first laboratory is to develop a functional simulator for a
portion of the MicroBlaze Instruction Set Architecture (ISA), namely, basic
arithmetic and logic operations, branches (delayed or otherwise), comparisons,
loads, and stores.



\section{Simulator architecture}

The simulator has a 5-stage pipeline microarchitecture without forwarding. The
driver routine (\verb|main|) repeatedly calls \verb|cpu_cycle|. The CPU halts
when all pipeline stages are disabled (\verb|cpu_halt|). Initially, only the
Instruction Fetch (IF) stage is enabled. The CPU starts with the Program Counter
(PC) at zero and with zeroed memory. The driver routine is responsible for
loading a program into memory (\verb|flash_memory|).

\begin{verbatim}
struct cpu_state_t {
    size_t total_cycles;
    size_t total_instructions;

    address_t pc;

    int halt;
    int has_delayed_branch;

    int if_stalls;
    int id_stall;

    word_t if_enable;
    word_t id_enable;
    word_t ex_enable;
    word_t mem_enable;
    word_t wb_enable;
} cpu_state;

struct msr_t {
    word_t c : 1;
    word_t i : 1;
} msr;

uint16_t rIMM;
\end{verbatim}




\subsection{Register File}

\begin{verbatim}
struct {
    word_t data;
    word_t in_ex : 1;
    word_t in_mem : 1;
    word_t in_wb : 1;
} registers[32];
\end{verbatim}

The register file, implemented in \verb|src/register.c|, holds the value of all
registers, and tracks their updates on the pipeline. On each clock cycle, the
routine \verb|register_clock| is called to advance the status bits in the
register file. Access is done through the \verb|register_read| and
\verb|register_write| routines.



\subsection{Memory}

\begin{verbatim}
word_t *mem = NULL;
size_t memsize = 0;
\end{verbatim}

\subsection{Instruction Fetch}

The IF stage (\verb|if_stage|), implemented in \verb|src/if_stage.c|, first proceeds to check if
there's a stall (\verb|cpu_state.id_stall != 0|) in the Instruction Decode (ID)
stage. If there is not, it checks whether it is stalled because of ongoing
branches (\verb|cpu_state.if_stalls > 0|).



\subsection{Instruction Decode}



\subsection{Execute}



\subsection{Memory}



\subsection{Write Back}



\section{Challenges}



\section{Conclusion}


\end{document}
