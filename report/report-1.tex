\documentclass[]{IEEEtran}
\usepackage[utf8]{inputenc}
\usepackage{listings}

\author{Rodrigo Lourenço, ist176133, group 9}
\title{Development of a simple RISC processor}

\begin{document}

\maketitle

\section{Introduction}

The goal for this first assignment is to develop a functional simulator for a
portion of the MicroBlaze Instruction Set Architecture (ISA), namely, basic
arithmetic and logic operations, branches (delayed or otherwise), comparisons,
loads, and stores.

\section{Simulator architecture}

The simulator has a 5-stage pipeline microarchitecture without forwarding. The
stages are, in order, Instruction Fetch (IF), Instruction Decode/Operand Fetch (ID),
Execute (EX), Memory (MEM), and Write Back (WB). The driver routine (\verb|main|)
repeatedly calls \verb|cpu_cycle|. The CPU halts when all pipeline stages are
disabled (\verb|cpu_halt|). Initially, only the IF stage is enabled. The CPU
starts with the Program Counter (PC) at zero and with zeroed memory. The driver
routine is responsible for initializing the CPU and the memory, and for loading
a program into memory. \verb|address_t|, \verb|instruction_t|, and \verb|word_t|
are aliases to C's \verb|uint32_t|.

\begin{verbatim}
struct cpu_state_t {
    size_t total_cycles;
    size_t total_instructions;

    address_t pc;

    int halt;
    int has_delayed_branch;

    int if_stalls;
    int id_stall;

    word_t if_enable;
    word_t id_enable;
    word_t ex_enable;
    word_t mem_enable;
    word_t wb_enable;
} cpu_state;

struct msr_t {
    word_t c : 1;
    word_t i : 1;
} msr;

uint16_t rIMM;
\end{verbatim}

The PC is stored in the \verb|cpu_state| variable. There are counters for
the number of cycles and instructions so we can calculate the instructions per
cycle at the end. The \verb|msr| variable holds the \verb|i| (immediate) and
\verb|c| (carry) flags. There's a \verb|rIMM| register to hold the upper 16 bits
of an instruction with an extented immediate. There are flags to control the
pipeline stages.

On each cycle (\verb|cpu_cycle|) the CPU invokes the pipeline stages backwards.
After executing a stage, by default it disables it, enables the next logical
stage, and clears its state so we can be sure there is no lingering data and/or
control that could trouble us.  The rationale is that if a stage has executed,
then it has data and/or control for stages further in the pipeline, and we have
no use for the old state. A stage may override the stage enable decision,
although this was not needed in practice.

(If the stages were not executed backwards then we would execute a whole
instruction per cycle since the results would be instantly propagated to the
following stage without waiting for the next cycle. This could be worked around,
e.g.\ using temporary space and copying back the stages' state, but it's
additional work and has little or no benefit.)

The exception is the IF stage, which is always enabled. It disables itself after
fetching a \verb|bri 0| instruction. This is because the Xilinx's compiler emits
this instruction on the libc's \verb|_exit| routine, which is eventually called
once \verb|main| terminates, and no instructions after are executed since it's
an infinite loop.


\subsection{Register File}

\begin{verbatim}
struct {
    word_t data;
    word_t in_ex : 1;
    word_t in_mem : 1;
    word_t in_wb : 1;
} registers[32];
\end{verbatim}

The register file, implemented in \verb|src/register.c|, holds the value of all
registers, and tracks their updates on the pipeline. On each clock cycle, the
routine \verb|register_clock| is called to advance the status bits in the
register file. Access is done through the \verb|register_read| and
\verb|register_write| routines. A register is marked for tracking using
\verb|register_mark_used|, and its status queried with \verb|register_in_used|.
The registers are zero initialized and writes to register 0 are ignored.


\subsection{Memory}

\begin{verbatim}
word_t *mem = NULL;
size_t memsize = 0;
\end{verbatim}

The memory, implemented in \verb|src/memory.c| is created with the \verb|init_memory|
routine, which receives the number of addressing bits. This is used to know how
large the memory is, and to mask off irrelevant bits so we have a cycling memory.

The memory is zero initialized, and can be flashed (e.g.\ from a file) using
\verb|flash_memory|. The simulator has a command-line option for specifying
endianness, which is used to convert the byte order transparently to the CPU,
which always executes in little endian. Memory access is done using the
\verb|memory_read| and \verb|memory_write| routines.


\subsection{IF stage}

\begin{verbatim}
struct if_state_t {
    int pc_sel;

    address_t branch_pc;
    address_t next_pc;
} if_state;
\end{verbatim}

The IF stage (\verb|if_stage|), implemented in \verb|src/if_stage.c|, first proceeds to check if
there's a stall (\verb|cpu_state.id_stall != 0|) in the ID
stage. If there is not, it checks whether it is stalled because of ongoing
branches (\verb|cpu_state.if_stalls > 0|) and the branch is not delayed. If none
of those conditions happen, then the new PC is chosen either from a branch, if
\verb|if_state.pc_sel == IF_SELPC_BRANCH|, or \verb|PC + 4|. This
PC is then used as an address when fetching an instruction from memory. If the
fetched instruction is \verb|bri 0|, the IF stage is disabled. The instruction
then moves on to the ID stage, along with its PC. The PC is always passed along
with the control and data signals so the CPU can keep track of what instruction
is at what pipeline stage.


\subsection{ID stage}

\begin{verbatim}
struct id_state_t {
    address_t pc;
    instruction_t instruction;
} id_state;
\end{verbatim}

This stage is a big, ugly switch statement which sets \verb|ex_state|'s and
\verb|cpu_state|'s fields according to the fetched instruction. Branches set
\verb|cpu_state.if_stalls| to 3, and delayed branches set
\verb|cpu_state.has_delayed_branch| to 1. If the operands to an instruction are
unavailable because they're further in the pipeline, the ID stage stalls by
setting \verb|cpu_state.id_stall| to 1, which halts the IF stage too. If they're
available, they're set to unavailable using \verb|register_mark_used|. The MSR's
immediate flag (\verb|msr.i|) is set to 0 unless an \verb|imm| instruction is
being decoded. Immediate extension occurs here, in the \verb|extend_immediate|
routine. One additional instruction is recognized, but unimplemented: \verb|mts|.
It is so because the Xilinx's compiler emits this instruction for the libc
initialization, so the simulator doesn't abort when seeing it.


\subsection{EX stage}

\begin{verbatim}
struct ex_state_t {
    address_t pc;

    word_t op_a, op_b, op_c;
    int alu_control, select_op_a;
    int carry_write_enable;
    int is_signed;
    int use_carry;

    int branch_enable;
    int branch_cond;

    address_t wb_dest_register;
    int wb_select_data;
    int wb_write_enable;
    word_t wb_data;

    int mem_access;
    int mem_mode;
    int mem_write_enable;
    word_t mem_data;
} ex_state;
\end{verbatim}

The fields starting with ``wb'' or ``mem'' are passed through to the MEM stage
untouched. Arithmetic is done depending on the control signal on
\verb|ex_state.alu_control|. \verb|ex_state.select_op_a| is a remnant of past
eons and will be removed in a future iteration. There's a flag to control if
operations are signed or not (\verb|ex_state.is_signed|), to use and/or keep the
carry flag (\verb|ex_state.use_carry| and \verb|!ex_state.carry_write_enable|,
respectively). If branching, calculate the target address, test the internal
zero and negative flags, and compare to the branch condition
(\verb|ex_state.branch_cond|). If the branch is to be taken, set the
\verb|if_state.branch_pc| and \verb|if_state.pc_sel| accordingly. The ALU's
result is placed in \verb|mem_state.alu_result|.


\subsection{MEM stage}

\begin{verbatim}
struct mem_state_t {
    address_t pc;

    address_t wb_dest_register;
    int wb_select_data;
    int wb_write_enable;
    word_t wb_data;

    word_t data;
    word_t alu_result;
    int write_enable;
    int memory_access;
    int mode;
} mem_state;
\end{verbatim}

The fields starting with ``wb'' are passed through to the WB stage untouched.
The address, coming from \verb|mem_state.alu_result|, is used to read from
memory. The read value is masked and shifted around according to whether or not
we're addressing bytes, half-words, or words, and stored in
\verb|wb_state.memory_out|. If \verb|mem_state.write_enable|, then data from the
ID stage, \verb|mem_state.data|, is masked and shifted around, OR'ed with the
relevant portions of the read data, and written to memory. This is to support
writes to half-words and bytes. An alternative implementation would read and
write from different memory banks.


\subsection{WB stage}

\begin{verbatim}
struct wb_state_t {
    address_t pc;
    word_t memory_out;
    word_t alu_result;
    word_t data;

    address_t dest_register;
    int select_data;
    int write_enable;
} wb_state;
\end{verbatim}

This is the logically simplest stage. We select what data to write
(\verb|wb_state.select_data|), among the PC, the MEM stage, the EX stage, and
the ID stage, to what register (\verb|wb_state.dest_register|) and store it if
\verb|wb_state.write_enable != 0|.


\section{Challenges}

A non-exhaustive list of bugs, logic or implementation, encountered:

\begin{itemize}
\item Forgetting to turn off the \verb|msr.i| flag after each instruction.
\item Right shifts. The C standard leaves implementation-defined whether the
right shift is arithmetic or not, so I had to implement it.
\item Undefined behaviour. Signed integer overflow, shifts by widths greater or
equal to the number of bits of the type. Mostly solved by judicious checking and
using unsigned values for everything.
\item Having no way to send data to the WB stage directly from the ID stage.
Broke the sign-extension instructions since these are performed in the ID stage,
and hence didn't go through the EX or MEM stages' data. Could have gone through
the EX stage as an addition with 0 and the \verb|ex_state.keep_carry| flag set.
\item Forgetting to deal with endianness in memory. \verb|lleHw ,odlro|.
\item Not paying attention to the compare instructions' operands. I had fun
debugging a problem where I implemented the \verb|cmp| and \verb|cmpu|
instructions by symmetry with the \verb|rsub[i][c][k]| family. I had not noticed
that the operands were swapped.
\item Opcode decoding. When testing for bitwise logical operations, the check
for \verb|andn| was wrong and turned the \verb|and| and \verb|xor| instructions
into \verb|andn| instructions.
\end{itemize}

The biggest difficulty was debugging the CPU. There are lots of things going on,
even in an implementation as simple as this one. I wrote some debugging code
which let me step through the CPU state, a cycle at a time. I also added options
to trace writes to memory and registers. A list of available options can be
found near the top of \verb|src/main.c|.


\section{Performance, and next steps}

It's slow. The test program ended with an IPC of $0.72$ (compiled with gcc 4.9.2
with -Ofast flag), which is hardly interesting. With forwarding that would
improve significantly, I believe. For the next assignment, I plan to implement
Tomasulo's algorithm, and severely rearchitect the simulator. But Tomasulo's
algorithm won't bring much to the table versus forwarding for instructions with
1 cycle of latency, so I plan to introduce delays in some heavier instructions.

Base plan:
\begin{itemize}
\item Implement Tomasulo's algorithm;
\item Compare the execution of the test program on the two microarchitectures;
\end{itemize}

If-there's-time-plan:
\begin{itemize}
\item Introduce 3 cycles of latency on multiply instructions. Add 3 levels of
cache , with increasing sizes, and latencies of 4, 10, and 40 cycles.
\item Implement 5-stage pipeline with forwarding;
\item Implement 5-stage multicycle;
\item Implement Tomasulo's algorithm with multiple issue.
\item Compare the execution of the test program on the five microarchitectures;
\end{itemize}


\section{Conclusion}

I have implemented a simple, functional simulator for a subset of the MicroBlaze
architecture. It was not a complicated endeavour, with the main problems being
caused by lack of attention in my part, since the MicroBlaze architecture is
quite simple, and very regular.

\end{document}
